\documentclass{article}
    % General document formatting
    \usepackage[margin=0.7in]{geometry}
    \usepackage{hyperref}
    \usepackage[parfill]{parskip}
    \usepackage[utf8]{inputenc}
    
    % Related to math
    \usepackage{amsmath,amssymb,amsfonts,amsthm}

\usepackage{graphicx}
\graphicspath{{figs/}}

\hypersetup{
    colorlinks=true,
    linkcolor=blue,
    filecolor=magenta,      
    urlcolor=cyan,
}

\begin{document}

\section{Introduction}

\subsection{Percolation}\label{subsec:Percolation}

Site percolation in complex networks can be stated by considering that each node of the network can be either \emph{occupied}, with probability $p$, or \emph{unoccupied}, with probability $1-p$. Only occupied nodes can be connected, thus links connecting at least one unoccupied node are also considered unoccupied. If $p=0$, the network is empty and if $p=1$, the original network is recovered. When the occupation probability is small, occupied nodes belong to different small-sized components, but above a critical value $p=p_c$, one of the components acquires an extensive size. At this point, it is said that the system percolates. The extensive component is known as the \emph{giant connected component} (GCC) and the critical point is referred to as the \emph{percolation threshold}.

Let $N$ be the linear size of the network and $N_1$ the size of the GCC. In the thermodynamic limit $N\rightarrow \infty$, percolation theory states that the relative size $S_1 = N_1/N$ follows the critical behavior
%
\begin{equation}
    S_1=
    \begin{cases}
    0 &  p <  p_c, \\
    a(p-p_c)^\beta &  p \ge p_c,
    \end{cases}
\end{equation}
%
where $a$ is a proportionality constant and $\beta>0$ is the critical exponent associated with $S_1$. The transition between the percolated and non-percolated state has been widely studied in statistical physics, and it has been shown to exhibit a continuous transition in many different network models. In this framework, $S_1$ is considered the order parameter of the transition. 

As it occurs in continuous transitions, other measures also manifest a critical behavior near the percolation threshold. One such measure is the average cluster size, which plays the role of susceptibility and is computed as 
%
\begin{equation}
    \langle s \rangle = \frac{\sum_{s}^{'} s^2 n_s(p)}{\sum_s^{'} s n_s(p)} 
\end{equation}
%
where $n_s(p)$ is the number of clusters of size $s$ per node and the primed sum excludes the GCC. At the critical point, $\langle s \rangle$ diverges in the thermodynamic limit as $\langle s \rangle  \sim |p-p_c|^{-\gamma}$, with $\gamma>0$. Also, $n_s(p)$ has its own critical behavior and close to $p_c$ it becomes very heterogeneous, being well described by the expression
%
\begin{equation} \label{eq:ns}
n_s(p) \sim s^{-\tau} e^{-s/s^*}.
\end{equation}
%
Here $s^*$ represents the characteristic cluster size, which scales as $s^* \sim |p-p_c|^{-1/\sigma}$. Then, at $p=p_c$ the number of clusters of size $s$ follows a power-law $n_s(p) \sim s^{-\tau}$. Finally, the correlation length $\xi$, defined as the geometrical length of a typical cluster, scales as $\xi \sim |p-p_c|^{-\nu}$, where $\nu>0$~\cite{StaufferBook}.


The theory of critical phenomena states that continuous transitions can be fully characterized by its critical exponents. If the same exponents are shared between two systems, they belong to the same universality class. In percolation only two exponents are independent, and the others can be derived using different scaling relations. For example, the exponent associated with the cluster size distribution can be obtained as \cite{StaufferBook}
%
\begin{equation}\label{eq:scaling_tau}
    \tau = 2 + \dfrac{\beta}{\gamma + \beta}.
\end{equation}
%
As $\beta$ and $\gamma$ are both positive, equation \ref{eq:scaling_tau} shows that $\tau \geq 2$. Another useful relation is given by \cite{BrankovBook, Fortunato2011ExplosiveGraphs}
%
\begin{equation} \label{eq:gamma_beta_nu}
    2\beta + \gamma = d\nu,
\end{equation}
%
where $d$ is the system dimension.

From a theoretical point of view, standard percolation and node removal are different processes~\cite{Cohen2000BreakdownAttack}. Percolation is an equilibrium reversible process, well described by the equilibrium statistical physics. 
On the other hand, node removal under specific attacks are irreversible  processes such as the evolving rules that turn out in explosive percolation transitions~\cite{DaCosta2014SolutionExponents}.
Being aware of this, we relate the percolation probability $p$ with a node removal procedure in which a fraction $f=1-p$ of nodes was removed. Using this relation we can apply the tools provided by percolation theory to the attack strategies previously described.

\subsection{Percolation transition on random spatial networks}

Percolation transition on random spatial networks has been largely studied \cite{Melchert2013,Becker2009,Norrenbrock2016a}. Although the location of the percolation threshold depends on the model studied, in general the universaillity class is the same as regular lattices. In particular, Norrenbrock, et al \cite{Norrenbrock2016FragmentationAttacks} study the percolation transiton for recalculated degree-based (RD) and betweenness-based (RB) attacks on four different models of spatial networks. They conclude that the RD attack belongs to the standard 2D percolation transition universality class. With respect to RB, they show that the percolation threshold is located at $f_c = 0$, but they do not arrive at a conclusion regarding other characteristics of the transition.


\subsection{Finite-size scaling analysis}


Finite-size scaling analysis is one of the most important tools in the study of  continuous phase transitions and in particular to obtain the associated critical exponents~\cite{Cho2010Finite-sizeTransitions, Fortunato2011ExplosiveGraphs, Zhu2017FiniteTransition}.
According to this theory, the divergence of the correlation length at the critical point implies that every variable of the system becomes scale-independent at this point. For a finite-size system of linear size $L$, this produces a scaling of the form
%
\begin{equation}
    X \sim L^{-\omega/\nu} F[(f-f_c) L^{1/\nu}],
\end{equation}
%
where $\omega$ is an exponent related to the variable $X$. For $f=f_c$, the variable behaves as $X \sim L^{-\omega/\nu}$. This relation holds asymptotically, i.e. in the limit $L \rightarrow \infty$ and $f \rightarrow f_c$, and it can be used to obtain the ratio $\omega/\nu$ by computing $X(f_c, L)$ for different system sizes. In addition, the plot of $L^{\omega/\nu}X$ as a function of $(f_c-f) L^{1/\nu}$ yields to the universal function $F$, which does not depend on $L$, so curves corresponding to different sizes collapse. 


In this work, we make use of two scaling relations. The first one is the scaling of the cluster relative sizes, which can be stated as  \cite{Zhu2017FiniteTransition}
%
\begin{equation}
\label{eq:comp-scaling}
S_i(f,L) \sim L^{-\beta /\nu} \tilde{S}_i[(f-f_c) L^{1/\nu}].
\end{equation}
%
 Here, the subscript $i=1,2, ...$ indicates the rank of each component, sorted by size in decreasing order. In particular, we will be interested in the order parameter $S_1$ and in the size of the second cluster $S_2L^d$. The second scaling relation involves the average cluster size and can be stated as
%
\begin{equation}
\label{eq:suscep-scaling}
\langle s \rangle(f,L)  \sim L^{\gamma / \nu} \tilde{S} [(f-f_c) L^{1/\nu}].
\end{equation}
%


For a finite-size system, the percolation threshold does not necesarly coincide with the corresponding value for $N\rightarrow \infty$. In general, the difference between these values presents a scaling in the form

\begin{equation} \label{eq:peak_pos_shift}
f_c(L) - f_c = b L^{-\lambda}.
\end{equation}

In standard percolation, as well as in many other models, $\lambda = 1/\nu$, but this is not always valid \cite{Li2012,Ziff2010ScalingLattice,Grassberger2011ExplosiveBehavior}. In some explosive percolation models, for example, small-sized systems present a differences between the two exponents due to large crossover sizes.

In a recently published article, Fan, et al. proposed a new method to analyze generalized percolation processes, based on the scaling of the largest jump in the order parameter during the process \cite{Fan2020}. Although the authors deal with bond percolation, the same analysis can be performed for site percolation. Based on this work, we define the \emph{gap}

\begin{equation} \label{eq:Delta}
\Delta = \dfrac{1}{L^d} \max_{t} \left[ N_1(t+1) - N_1(t) \right],
\end{equation} 

where $N_1(t)$ is the size of the largest cluster after $t$ nodes are removed from the network. According to \cite{Fan2020}, $\Delta\sim L^{-\beta/\nu}$. We also define $t_c$ as the number of nodes removed such that the maximum in \ref{eq:Delta} is attained and the \emph{gap threshold} $r_c(L) = t_c/L^d$. For a finite-size system, $r_c(L)$ may differ from the finite-size percolation threshold $f_c(L)$, but in the thermodynamic limit $f_c = r_c(\infty)$. %Finally, the percolation strength is defined as $N_1(t_c)$. 
According to \cite{Fan2020}, these quantities, averaged over realizations, scale as 

\begin{align}
\Delta&\sim L^{-\beta/\nu} \label{eq:fan_Delta}\\
r_c(L) - r_c(\infty) &\sim L^{-1/\nu_1} \label{eq:fan_rc}
%N_1(L) &\sim L^{-d_f}. \label{eq:fan_N1}
\end{align}

Also, the fluctuations for these quantities present a similar scaling 

\begin{align}
\chi_{\Delta} &\sim L^{-\beta/\nu} \label{eq:fan_chiDelta} \\
\chi_{r_c} &\sim L^{-1/\nu} \label{eq:fan_chirc}
%\chi_{N_1} &\sim L^{-d_f}. \label{eq:fan_chiN1}
\end{align}

Note that, in a similar way as it happens in Equation \ref{eq:peak_pos_shift}, the exponent $\nu_1$ does not coincide in general with the correlation length critical exponent $\nu$.

%In equilibrium thermodynamics, first-order transitions do not present a divergence of the correlation length and thus, do not exhibit critical behaviour. For out-of-equilibrium systems, this is not the general case. Several examples exhist where both a discontinuity of the order parameter and criticallity coexist [ADD REFS]. 

For first-order transitions, the order parameter shows a discontinuous jump at the percolation threshold, which implies $\beta = 0$. Besides, the susceptibility scales with system dimension as $\gamma / \nu = d$ \cite{Binder1981,Binder1984,Cho2009}.

\section{Results}

\subsection{Recalculated-Betweenness attack on DT networks}

In Figure \ref{fig:collapse_RB_DT} we show different observables of the phase transition for different sizes, averaged over $10^5$ realizations. In (a), the order parameter $S_1$ is shown. In the begining of the attack, the network remains fully connected, but after a given number of nodes are removed, it begins to break down, presenting an elbow-like shape. As the systems become larger, the position of the elbow shifts towards $f = 0$, meaning that a smaller fraction of nodes needs to be removed in order to break the network. This result is coincident with previous study of the transition \cite{Norrenbrock2016FragmentationAttacks}. Figures (b) and (c) show the peak in the susceptibility $\langle s \rangle$ and second largest clsuter $N_2=S_2L^2$. In both cases, the peak also moves towards $f = 0$ as the system size increases. 

\begin{figure}[h]
\centering
\includegraphics[scale=0.25]{collapse_RB_DT.pdf}
\caption{\label{fig:collapse_RB_DT} (Upper panels) Relative size of the giant component (left), average finite-component size (center) and size of the second largest component for the DT network as a function of the fraction of nodes removed under the RB attack. (Lower panels) Collapse of the curves.}
\end{figure}


Following the scaling ansatz \ref{eq:comp-scaling} and \ref{eq:suscep-scaling}, we computed the exponent ratios $\beta / \nu$ and $\gamma / \nu$ by fittig the size of the peaks of $S_2$ and $\langle s \rangle$ for different system sizes. The values obtained are consistent with a first-order transition. 

%Figures (d-f) show the collapse of the curves using the scaling ansatz and the computed exponents. 


If the percolation threshold is indeed $f_c = 0$, then the finite-size threshold $f_c(L)$ should exhibit a power-law behaviour in accordance with Equation \ref{eq:peak_pos_shift}. To see this, we computed $f_c(L)$ for different sizes and plotted against $L$. We define $f_c(L)$ using two different criteria, namely as the peak position for either $\langle s \rangle$ and $S_2$. In Figure (\ref{fig:exponents}b) we show that this is indeed the case, and also obtain consistent values for the scaling exponent.


\begin{figure}[h]
\centering
\includegraphics[scale=0.25]{beta_gamma_nu_scaling_RB_DT.pdf}
\caption{\label{fig:exponents}(Left) Scaling for the peaks of $\langle s \rangle$ and $S_2 L^2$ as a function of the linear size $L$. The exponents obtained are consistent with $\beta =0$, $\gamma = d\nu$, corresponding to a first-order transition. (Right) Scaling for the shifting of the finite-size percolation threshold, defined as the position of the peak of the corresponding measure. Both shiftings show a similar power-law scaling, with consistent exponents.}
\end{figure}





\begin{figure}[h]
\centering
\includegraphics[scale=0.25]{gap_exp_scaling_RB_DT.pdf}
\caption{\label{fig:gap_exp} Estimation of the critical exponents using the gap scaling proposed by Fan, et al. in \cite{Fan2020}.}
\end{figure}


\subsection{Attacks with approximate betweenness}

In this section we compare the RB attack with attacks using $\ell$-betweenness, i.e, computing betweenness using only paths up to length $\ell$. 

In Figure (\ref{fig:RBl_attacks}) we show the performance of attacks based on approximate betweenness for different approximation lengths $\ell$ on a network with $L^2 = 1024$. As expected, longer values of $\ell$ give a better attack, shifting the percolation transition to the left. For sufficiently large $\ell$, the curves become indistinguishable from the RB attack. This indicates that, for each size $L$, a finite approximation length $\ell^*(L)$ exists such that for $\ell \geq \ell^*(L)$ all RB$\ell$ attacks perform as well as RB. In other words, all the relevant information from the system is available for the dismantling process for such approximation length. 

\begin{figure}[h]
\centering
\includegraphics[scale=0.25]{order_par_and_susceptibility_RBl_DT.pdf}
\caption{\label{fig:RBl_attacks} Performance of RB$\ell$ attacks on a DT network of size $L^2=1024$. For this size, $\ell^*=25$.}
\end{figure}

In Figure (\ref{fig:rc_per_cutoff}a) we show the behaviour of the gap threshold $r_c(L)$ for RB$\ell$ attacks with different values of $\ell$. As it can be seen, the attack improves as $\ell$ increases, until the gap threshold equals the corresponding value for the full RB attack (dashed lines). For any $\ell\geq \ell^*$, $r_c^{(\mathrm{RB}\ell)}(L)$ remains constant. We define $\ell^*(L)$ as the lower $\ell$ such that $|r_c^{(\mathrm{RB}\ell)}(L) - r_c^{(\mathrm{RB})}(L)| < c$, and chose $c = 0.01$ (the definition is robust for different values of the threshold $c$). 
Figure (\ref{fig:rc_per_cutoff}a) shows the scaling of $\ell^*(L)$ against $L$. It can be seen that $\ell^* \sim L^{\alpha}$, with $\alpha \approx 0.76$. To compare this exponent with network properties, we computed the network average shortest-path length and diameter for different sizes. Both meausures display also a power-law scaling with the system, with exponent $0.79\pm 0.01$ (grey lines in the figure). For all sizes studied, $\ell^*$ is greater that the diameter, and although its scaling exponent is slightly lower than the one corresponding to the diameter, it is hard to say if the two curves will cross for larger systems.  

\begin{figure}[h]
\centering
\includegraphics[scale=0.3]{peak_shifting_RBl_rc_DT.pdf}
\caption{\label{fig:rc_per_cutoff} (Left) Shift of the position of the percolation threshold for the RB attacks with cutoffs. Dashed lines indicate the corresponding valu for the RB attack. (Right) Scaling of the $\ell^*$ with the system linear size. Here, $\ell^*$ is defined as the minimum $\ell$ such that the percolation threshold  for the RB-$\ell^*$ differs from the full RB attack in less than $1\%$. The grey lines correspond to the diameter (solid) and average shortest-path length (dot-dashed) for the DT network prior to the attack, which scale both as $\sim L^{\omega}$, with $\omega = 0.79\pm 0.01$.}
\end{figure}


\subsection{Breaking nodes and fractals}

In Figure \ref{fig:attack_draw} we show the network status after the first $t_c$ nodes are removed (i.e., right after the largest relative jump of the giant cluster takes place). We compare the 2D Lattice (left panels) with the DT (right panels), and different attacks. As the figure shows, the RB attack cuts the lattice by the main diagonal, and breaks the DT in two extensive components. Thus, we call this set of nodes the \emph{breaking nodes}. 

\begin{figure}
\centering
\includegraphics[scale=0.3]{Lattice_vs_DT_L128_seed00000.png}
\caption{\label{fig:attack_draw} Network status at $f_c$ for RB $(N = 128^2)$. After removing the green nodes, the network is splitted in two extensive clusters. The largest cluster correspond to the orange nodes and the second largest cluster to the purple nodes. The rest of the nodes are plotted in grey. For the Lattice, RB traces a diagonal and break the network in two clusters of almost the same size. For DT, the algorithm starts by removing the high-betweennesss nodes located at the periphery and then it finds a high betweenness ``percolating path'' across  the network. Although in the latter case the network is also splitted in two extensive components, there is a significant difference in size between the two largest clusters. }
\end{figure}


To characterize the geometry of the \emph{breaking nodes}, we estimated its fractal dimension using two common methods [CITE: Stanley book]. The first method, known as the box-counting method, consists in covering the space by non-overlapping boxes of linear size $\delta \leq L$ and count the number $N_B$  of such boxes containing at least one node belonging to the set. The procedure is repeated for different values of $\delta$. Fractal objects are expected to follow a relation $N_B(\delta) \sim \delta^{-d_B}$, where $d_B$, known as the box-counting dimension, is its characteristic length. For the second method, we select a random element of the set and consider a square window of linear size $l$ centered at the element. We count the number of elements of the set in the window and repeat the procedure for different values of $l$. We then average these values taking several seeds. For each value of $l$, the mass $M(l)$ is computed as the average over seed elements of the number of nodes in the window, and is expected to follow the scaling $M(l)\sim l^{d_f}$, where $d_f$ is the fractal dimension of the set. In Figure \ref{fig:Delta_Lattice} we show both estimations for the DT network and for the 2D Lattice. Coincident with Figure \ref{fig:attack_draw}, the fractal dimension for the set of removed nodes from the Lattice is equal to one, as the nodes are roughly arranged over a straight line. For the case of DT, a non-trivial scaling is seen, with a fractal dimension greater than one. Note that the values obtained using both methods are consistent ($d_B = 1.45\pm 0.16$ and $d_f = 1.43\pm 0.06$).

There is a third way of estimating the fractal dimension of the set, which consists in using Equation \ref{eq:fan_rc}. According to this equation, the mass of the set scales as $M\sim N^{1-1/\nu_1} = L^{d_{f_2}}$, where $d_{f_2} := 2(1-1/\nu_1)$. Taking the value estimated for $\nu_1$, one obtains $d_{f_2} = 1.44$.

The breaking nodes for the case of RB on the Lattice consist mainly in one of the main diagonals. Thus, if we take the spanning subgraph with vertex set equal to the breaking nodes, we have mainly isolated nodes. In contrast, when applying the same attack to the DT, the spanning subgraph corresponding to the breaking nodes has a large connected component. We call this component the \emph{backbone} of the breaking nodes. In Figure \ref{fig:fractal_dimension}, we measure the fractal dimension of this set. We can see that, in this case, the fractal dimension of the set is close to one, indicating that the backbone is not fractal, but a conventional curve. 

As it can be infered from Figure \ref{fig:attack_draw}, the backbone itself is sufficient to break the network. We corroborated this proposition by measuring the relative size of the largest cluster after removing only the backbone nodes. As Figure \ref{fig:Sgcc_backbone} shows, $S_1$ has a comparable size after removing the backbone with the result of removing all the breaking nodes. 
In terms of attack performance, we can say that, although the RB attack is very efficient in breaking the network (the percolation threshold is $r_c = 0$ in the thermodynamic limit), is not optimal, as the number of nodes that the attack removes scales more that linearly. But, if instead of removing the breaking nodes, they are only algorithmically identified, one can chose to remove only the backbone, disconnecting the network with an efficiency now close to optimal.

\begin{figure}
\centering
\includegraphics[scale=0.3]{df_L128_BtwU.pdf}
\includegraphics[scale=0.3]{df_L256_BtwU_conn.pdf}
\caption{\label{fig:fractal_dimension} (Upper panels) Fractal dimension of the breaking nodes for networks of size $N=128^2$. Left: Box-counting dimension; Right: estimation based on random seed sampling. (Blue) DT; (Orange) 2D Lattice. (Lower panels) Fractal dimension of the backbone for a network of type DT of size $N=256^2$. Left: Box-counting dimension; Right: estimation based on random seed sampling.}
\end{figure}


\bibliographystyle{unsrt}
\bibliography{library}


\end{document}

